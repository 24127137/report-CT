\section{Abstract}
This article presents an overview of recommendation systems, focusing on their components, classification, main techniques, applications, and a comparison of different approaches. 
Recommendation systems are essential tools in various domains, including e-commerce, entertainment, and social media, as they help users discover relevant content based on their preferences and behaviors.
\section{Introduction}
\subsection{About recommendation system}
What is it? \\
Recommendation systems address the issue of information overload. The systems personalize user experiences by suggesting items such as products, services, or content that align with individual preferences. Therefore, enhance user experiences and 
increase engagement and satisfaction.
Recommendation systems are divided into three types: Content-based filtering, Collaborative filtering, and Hybrid methods. Recommendation systems are now have the knowledge-based 

\section{ Analysis of Personalized Recommender Systems}

\subsection{Collaborative Filtering (CF)}
Driven by user profiles or historical interactions
\subsubsection{Memory-Based CF}
is simple and popular but
it faces challenges such as sparsity in the
interaction matrix and computational
complexity in extensive user or item
domains. \\
Solution is to transform
high-dimensional sparse vectors into
low-dimensional dense ones, e.g prefs2vec by Valcarece et al. 
\subsubsection{Model-Based CF}
Use mathematical model to predict the user-item relationships.
For example: Neural Network Matrix Factorization (NNMF) and Neural Collaborative Filtering (NCF) demonstrating the power of multilayer perceptrons (MLPs) for latent feature learning. 
Able to capture complex relationships.
\subsubsection{Context-Aware-Based CF}
enhances recommendation quality by incorporating contextual information.
provide more relevant and timely suggestions by using dynamic information like time, location, weather.
\subsection{Content-Based Systems}
Content-Based recommender systems treat recommendation as a user-specific classification problem. The system learns a classifier for a user based on the features of items they have previously rated positively. The model then uses this classifier to recommend other items with similar features.
\subsection{Knowledge-Based Systems}
Knowledge-Based recommender systems are uniquely suited for high-consideration, infrequent purchase domains (e.g., automobiles, real estate) where traditional methods falter.
This category employs two main methodologies:
\begin{itemize}
    \item Knowledge Graph Embedding (KGE):
    This method encodes external knowledge from sources like knowledge graphs into higher-level, dense vector representations. 
    \item Path-Based RS: This method leverages connectivity patterns, known as meta-paths, within a knowledge graph. By analyzing these paths, the system can evaluate the semantic similarity between entities
\end{itemize}

\subsection{LLM-Based Systems}

\subsection{Hybrid Systems}

\section{Challenges in Recommender System Design}
\subsection{System Robustness}
\subsection{Scalability}
\subsection{Data Bias}
\section{Similarity Measures}

Similarity measures are fundamental to certain recommendation approaches, particularly memory-based collaborative filtering, where they are used to identify neighboring users or items. The most fundamental of these are the Pearson Correlation Coefficient and Cosine Similarity.
Pearson Correlation Coefficient (PCC): This measures the linear correlation between the ratings of two users (u and v) on their co-rated items (Iu,v), defined as:
Cosine Similarity (COS): This measures the cosine of the angle between two user rating vectors in the item space, defined 
\section{Conclusion}
