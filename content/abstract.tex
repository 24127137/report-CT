\section{Introduction}
\subsection{About recommendation system}
What is it? \\
A recommendation system is a subclass of information filtering systems that seek to predict the "rating"
\subsection{Components of recommendation system}
The following are the specific technical objectives that need to be built to address the problems of loneliness and mismatched interests in travel:

\begin{enumerate}
    \item \textbf{Candidate generation}
Là bước đầu tiên nhằm tạo ra một tập hợp nhỏ các mục (items) có khả năng cao được người dùng quan tâm, từ hàng triệu hoặc hàng tỷ mục có sẵn.

    \item \textbf{Scoring}
Các mục ứng viên được đánh giá và chấm điểm bằng mô hình học máy để dự đoán mức độ quan tâm hoặc tương tác của người dùng đối với từng mục.

    \item \textbf{Re-ranking}
Sắp xếp lại danh sách các mục đã được chấm điểm, áp dụng các quy tắc hoặc mục tiêu bổ sung (ví dụ: tính đa dạng, tính mới, hoặc các ràng buộc nghiệp vụ) trước khi hiển thị kết quả cuối cùng cho người dùng.

\end{enumerate}

\subsection{Classification and Main Techniques}
kỹ thuật phổ biến được sử dụng để xây dựng hệ thống gợi ý, đặc biệt là trong bước Tạo Ứng viên (Candidate Generation):
\begin{enumerate}
    \item Lọc dựa trên Nội dung (Content-based filtering): Gợi ý các mục tương tự với các mục mà người dùng đã thích trong quá khứ.
    \item Lọc Cộng tác và Phân rã Ma trận (Collaborative Filtering and Matrix Factorization): Gợi ý các mục mà những người dùng có sở thích tương tự đã thích. Phân rã Ma trận là một kỹ thuật toán học phổ biến để thực hiện Lọc Cộng tác.
    \item Học Sâu (Deep Learning): Sử dụng mạng nơ-ron sâu để học các biểu diễn phức tạp của người dùng và mục, từ đó cải thiện độ chính xác của gợi ý.
\end{enumerate}
\subsection{Applications}
Ứng dụng phổ biến: Gợi ý sản phẩm trên các sàn thương mại điện tử (Amazon, Shopee), gợi ý phim/nhạc (Netflix, Spotify, YouTube), gợi ý tin tức hoặc bài viết (Google News, Facebook feed).
\subsection{Comparison}
